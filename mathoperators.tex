\usepackage[utf8]{inputenc}
\usepackage{amsthm}
\usepackage{amsfonts}
\usepackage{amssymb}
\usepackage{xcolor}
\usepackage{graphicx}
\usepackage{tikz}
\usetikzlibrary{babel,arrows,positioning,chains,matrix,scopes,cd,quotes,calc,decorations.pathmorphing}
\usepackage{caption}
\usepackage{subcaption}
\usepackage{algorithmic}
\usepackage[portuguese, ruled, lined]{algorithm2e}
\usepackage{setspace}
\usepackage[T1]{fontenc}
\usepackage{csquotes}

%quiver
\tikzset{curve/.style={settings={#1},to path={(\tikztostart)
    .. controls ($(\tikztostart)!\pv{pos}!(\tikztotarget)!\pv{height}!270:(\tikztotarget)$)
    and ($(\tikztostart)!1-\pv{pos}!(\tikztotarget)!\pv{height}!270:(\tikztotarget)$)
    .. (\tikztotarget)\tikztonodes}},
    settings/.code={\tikzset{quiver/.cd,#1}
        \def\pv##1{\pgfkeysvalueof{/tikz/quiver/##1}}},
    quiver/.cd,pos/.initial=0.35,height/.initial=0}

% TikZ arrowhead/tail styles.
\tikzset{tail reversed/.code={\pgfsetarrowsstart{tikzcd to}}}
\tikzset{2tail/.code={\pgfsetarrowsstart{Implies[reversed]}}}
\tikzset{2tail reversed/.code={\pgfsetarrowsstart{Implies}}}
% TikZ arrow styles.
\tikzset{no body/.style={/tikz/dash pattern=on 0 off 1mm}}

\newtheorem{definition}{Definição}[section]
\newtheorem{proposition}[definition]{Proposição}
\newtheorem{lemma}[definition]{Lema}
\newtheorem{axiom}[definition]{Axioma}
\newtheorem{corollary}[definition]{Corolário}
\newtheorem{theorem}[definition]{Teorema}
\newtheorem{distribution}[definition]{Distribuição}
\newtheorem{example}[definition]{Exemplo}

\newcommand{\highlight}[1]{{\color{red3} #1}}

\DeclareMathOperator{\freeab}{Free_{Ab}}
\DeclareMathOperator{\Top}{Top}
\DeclareMathOperator{\Ab}{Ab}
\DeclareMathOperator{\Grp}{Grp}
\DeclareMathOperator{\im}{Im}
\DeclareMathOperator{\module}{Mod}
\DeclareMathOperator{\Int}{Int}
\DeclareMathOperator{\coker}{coker}
\DeclareMathOperator{\chain}{Ch}
\DeclareMathOperator{\Hom}{Hom}
\DeclareMathOperator{\sign}{sign}
\DeclareMathOperator*{\argmax}{arg\,max}
\DeclareMathOperator*{\argmin}{arg\,min}
\DeclareMathOperator{\Alt}{Alt}
\DeclareMathOperator{\sgn}{sgn}
\DeclareMathOperator{\supp}{supp}
\DeclareMathOperator{\cl}{cl}
\DeclareMathOperator{\End}{End}
\DeclareMathOperator{\grad}{grad}
\DeclareMathOperator{\Exp}{Exp}
\DeclareMathOperator{\codim}{codim}
\DeclareMathOperator{\Ricci}{Ricci}
\DeclareMathOperator{\arccosh}{arccosh}
\DeclareMathOperator{\gyr}{gyr}
\DeclareMathOperator{\ath}{arctanh}
\DeclareMathOperator{\Prob}{Prob}
\DeclareMathOperator{\tr}{tr}
\DeclareMathOperator{\Id}{Id}
\DeclareMathOperator{\GL}{GL}
\DeclareMathOperator{\adj}{adj}
\DeclareMathOperator{\On}{O}
\DeclareMathOperator{\SO}{SO}
\DeclareMathOperator{\Un}{U}
\DeclareMathOperator{\SU}{SU}
\DeclareMathOperator{\Graf}{Graf}
\DeclareMathOperator{\Real}{Re}
\DeclareMathOperator{\Imag}{Im}
\DeclareMathOperator{\Bij}{Bij}

\newcommand{\Mod}[1]{$\module_{#1}$}
\newcommand{\Chain}[1]{$\chain(#1)$}

\newcommand{\openset}[0]{{\phantom{}\subset}{\circ}\phantom{.}}

\def\arrvline{\hfil\kern\arraycolsep\vline\kern-\arraycolsep\hfilneg}